\documentclass{article}
\usepackage[english, greek]{babel}
\usepackage[utf8]{inputenc}
\usepackage{titlesec}
\usepackage{amsmath}

\titleformat{\section}
  {\normalfont\fontsize{12}{15}\bfseries}{\thesection}{1em}{}

\title{Γραφικά Υπολογιστών και Συστήματα Αλληλεπίδρασης\\ ΠΡΟΓΡΑΜΜΑΤΙΣΤΙΚΗ ΑΣΚΗΣΗ 2}
\author{Σιρινιάν Αράμ Εμμανουήλ \\ AM: 2537\\ \selectlanguage{english}ae.sirinian@gmail.com\selectlanguage{greek}}
\date{Ιανουάριος 2019}

\renewcommand{\thesection}{}
\renewcommand{\thesubsection}{}

\begin{document}

\maketitle

\section{Τα ερωτήματα που υλοποιήθηκαν}

Τα ερωτήματα της προγραμματιστικής ασκήσεις που υλοποιήθηκαν είναι τα:
\selectlanguage{english}
(1), (2), (3), (4), (5), (7).
\selectlanguage{greek}
Δηλαδή όλα εκτός από το
\selectlanguage{english}
(6)
\selectlanguage{greek}
και τα
\selectlanguage{english}
Bonus
\selectlanguage{greek}
ερωτήματα. Πέρα από τα ερωτήματα που δεν υλοποιήθηκαν
το πρόγραμα θα πρέπει να δουλεύει με τον τρόπο που περιγράφει η εκφώνηση.

\section{Μερικές λεπτομέρειες υλοποίησης}

Διαφοροποιήσεις από την εκφώνηση δεν υπάρχουν στα ερωτήματα.
Η υλοποίηση έγινε με το
\selectlanguage{english}
Unity 2018.3.
\selectlanguage{greek}
Ότι δεν έχει γίνει
\selectlanguage{english}
Manual,
\selectlanguage{greek}
 είναι γραμμένο σε ένα αρχείο
\selectlanguage{english}
 C\#
\selectlanguage{greek}
 μαζεμένο όλο. Το αρχείο
\selectlanguage{english}
 graphics-project-2/Assets/Scripts/ViewModel.cs
\selectlanguage{greek}
 είναι πολύ κακογραμμένο (συγνώμη για αυτο).
 Στην εκφώνηση όμως δεν αναφέρονται διάφορες λεπτομέρειες υλοποίησης
. Μία από αυτές τις λεπτομέρειες είναι το μέγεθος των κύβων,
 του παίκτη και των κυλίνδρων (κύλινδροι στο παιχνίδι δεν υπάρχουν). Αυτά
 τα μεγέθη είναι σταθερά και και δεν θα πρέπει να πειραχτούν με κάποιο
 τρόπο γιατί πολύ πιθανό αυτό να προκαλέσει πολλά
\selectlanguage{english}
Bugs.
\selectlanguage{greek}
 Μία ακόμα λεπτομέρεια
 που δεν αναφέρεται στην εκπόνηση είναι στο ερώτημα 2. Όταν κοίτα ένα
 κυβάκι μπροστά του μπορεί να πηδήξει πάνω του και να πάρει 10 βαθμούς
 αυτή την ενέργεια μπορεί να την επαναλάβει όσες φορές θέλει στο ίδιο
 κυβάκι. Επίσης όταν ο παίκτης πέφτει στο κενό επανατοποθετείται στην
 θέση που πατούσε ακριβώς την στιγμή πριν πέσει, σε περίπτωση που
 εξαφανίζει το κυβάκι κάτω από αυτόν και πέσει στο κενό θα συνεχίσει να
 πέφτει μέχρι να τελειώσουν οι ζωές του. Μία ακόμα λεπτομέρεια που αξίζει
 να παρατηρηθεί είναι όταν ο παίκτης βρίσκεται σε ένα επίπεδο και πέφτει
 κάτω πηδώντας, με τον τρόπο που υλοποιήθηκε το παιχνίδι ο παίκτης πολύ
 πιθανό να χάσει πόντος όσους θα έχανε αν έπεφτε χωρίς να πηδήξει από ένα
 επίπεδο πάνω. Στο ερώτημα 3 γίνεται λόγος για 2
\selectlanguage{english}
 spotlight,
\selectlanguage{greek}
 πέρα από τα
 σημεία που πρέπει να βρίσκονται στους άξονες
\selectlanguage{english}
 z \& x
\selectlanguage{greek}
 δεν γίνεται άλλος λόγος
 και αυτές τις παραμέτρους καθορίζονται με ένα δικό μου τρόπο. Στο ερώτημα
 4 δεν υπάρχουν λεπτομέρειες για τις αποστάσεις από τις οποίες μπορεί ο παίκτης
 να κοιτάει ένα κυβάκι και να παίρνει ένα εικονικό κύβο. Οι αποστάσεις είναι
 σχετικά μικρές ώστε να μην μπορεί ο παίκτης τα παίρνει εικονικά κυβάκια πάνω
 από μία μικρή και λογική απόσταση. Ο παίκτης μπορεί να τοποθετήσει έναν κύβο
 σε τέσσερις δυνατές τοποθεσίες. Όταν ο παίκτης ανεβαίνει στο επίπεδο
\selectlanguage{english}
N
\selectlanguage{greek}
 παίρνει
 τους αντίστοιχους βαθμούς και ζωές μόνο μία φορά σε όλο το παιχνίδι. Τέλος οταν ολες οι ζωές του
 παίκτη τελειώσουν το παιχνίδι εξαφανίζει τον παίκτη και συνεχίζει να τρέχει, αν
 θέλει ο χρήστης να ξαναρχίσει την προσπάθεια πρέπει να κλείσεις το πρόγραμμα
 και να το ξανανοίξει.

\selectlanguage{english}
\section{Bugs}
\selectlanguage{greek}

\selectlanguage{english}
1 Bug
\selectlanguage{greek}
που μπορεί να παρουσιαστεί είναι στους βαθμούς
που χάνει ο παίκτης όταν πέφτει επίπεδα. Αυτό αν
συμβαίναι οφείλεται στην ταχύτητα του υπολογιστή γιατί
ο τρόπος που υλοποιήθηκε βασίζεται στη μέτρηση του
ύψους τις πτώσεις του παίκτη που υπολογίζεται με μετρήσεις
ανά
\selectlanguage{english}
Update.
\selectlanguage{greek}

\section{Εκτέλεση}

Για να εκτελέσετε το πρόγραμμα πρέπει να κάνετε
\selectlanguage{english}
unzipe
\selectlanguage{greek}
τον φάκελο με τον κώδικα και τα αρχεία του
\selectlanguage{english}
unity: graphics-project-2.7z
\selectlanguage{greek}
Εκτελέστε το μέσω του
\selectlanguage{english}
Unity.
\selectlanguage{greek}

\section{Ομάδα}

Στην ομάδα είναι μόνο ένα άτομο (Σιρινιάν Αράμ Εμμανουήλ, 2537)

\end{document}
