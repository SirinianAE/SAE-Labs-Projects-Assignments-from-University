\documentclass{article}
\usepackage[english, greek]{babel}
\usepackage[utf8]{inputenc}
\usepackage{titlesec}
\usepackage{amsmath}

\titleformat{\section}
  {\normalfont\fontsize{12}{15}\bfseries}{\thesection}{1em}{}

\title{Γραφικά Υπολογιστών και Συστήματα Αλληλεπίδρασης\\ ΠΡΟΓΡΑΜΜΑΤΙΣΤΙΚΗ ΑΣΚΗΣΗ 1}
\author{Σιρινιάν Αράμ Εμμανουήλ \\ AM: 2537\\ \selectlanguage{english}ae.sirinian@gmail.com\selectlanguage{greek}}
\date{Δεκέμβριος 2018}

\renewcommand{\thesection}{}
\renewcommand{\thesubsection}{}

\begin{document}

\maketitle

\section{Τα ερωτήματα που υλοποιήθηκαν}

Τα ερωτήματα της προγραμματιστικής ασκήσεις που υλοποιήθηκαν είναι τα:
\selectlanguage{english}
(i), (ii), (iii), (iv), (v), (vi).
\selectlanguage{greek}
Δηλαδή όλα εκτός από το
\selectlanguage{english}
EXTRUDE
\selectlanguage{greek}
και τα
\selectlanguage{english}
Bonus
\selectlanguage{greek}
ερωτήματα. Πέρα από την εντολή
\selectlanguage{english}
EXTRUDE
\selectlanguage{greek}
που δεν έχει υλοποιηθεί θα πρέπει να δουλεύει με τον τρόπο που περιγράφει η εκφώνηση.

\section{Διαφοροποίηση από την εκφώνηση}

Μια μικρή διαφοροποίηση από την εκφώνηση αποτελεί το
\selectlanguage{english}
(ii)
\selectlanguage{greek}
ερώτημα όπου όταν θέλει ο χρήστης να ολοκληρώσει τη δημιουργία μιας
κλειστής πολυγωνικής γραμμής πατά το \underline{μεσαίο πλήκτρο του ποντικιού}
και τότε επιπρόσθετα η τελευταία κορυφή θα ενωθεί αυτόματα με την πρώτη.

\selectlanguage{english}
\section{Bugs}
\selectlanguage{greek}

Ένα μικρό
\selectlanguage{english}
bug
\selectlanguage{greek}
του προγράμματος είναι πως δεν μπορεί να αλλάξει το πρώτο
 πολύγωνο που φτιάχνει ο χρήστης χρώματα
\selectlanguage{english}
(LINE\_COLOR, FILL\_COLOR),
\selectlanguage{greek}
αλλά για όλα τα υπόλοιπα πολύγωνα που μπορεί να θέλει ο χρήστης να
δημιουργήσει η λειτουργία αυτή πρέπει να δουλεύει σωστά.


Ένα δεύτερο πρόβλημα με το πρόγραμμα παρατηρείται όταν
σε ένα πολύγωνο ύστερα από την λειτουργεία
\selectlanguage{english}
CLIPPING
\selectlanguage{greek}
πρέπει να δημιουργηθούν περισσότερα του ενός πολύγωνα.

\section{Εκτέλεση}

Γειά να εκτελέσετε το πρόγραμμα πρέπει να κάνετε
\selectlanguage{english}
make
\selectlanguage{greek}
από το
\selectlanguage{english}
project1\_Graphics/src/
\selectlanguage{greek}
φάκελο. Τέλος εκτελέστε το
\selectlanguage{english}
project1\_Graphics/src/project1
\selectlanguage{greek}
που θα δημιουργηθεί.
Επειδή το
\selectlanguage{english}
turnin
\selectlanguage{greek}
έχει μέγιστο όριο αρχείων 20 ο φάκελος με τα αρχεία βρίσκεται στο
\selectlanguage{english}
project1\_Graphics.tar.gz
\selectlanguage{greek}
 αρχείο.

\section{Ομάδα}

Στην ομάδα είναι μόνο ένα άτομο (Σιρινιάν Αράμ Εμμανουήλ, 2537)

\end{document}
